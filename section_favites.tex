%% START FAVITES CHAPTER
\chapter{\favitestitle}
\label{chap:favites}
\newpage

\textit{Motivation} --- The ability to simulate epidemics as a function of model parameters allows insights that are unobtainable from real datasets. Further, reconstructing transmission networks for fast-evolving viruses like \gls{hiv} may have the potential to greatly enhance epidemic intervention, but transmission network reconstruction methods have been inadequately studied, largely because it is difficult to obtain ``truth'' sets on which to test them and properly measure their performance.

\textit{Results} --- We introduce FAVITES, a robust framework for simulating realistic datasets for epidemics that are caused by fast-evolving pathogens like \gls{hiv}. FAVITES creates a generative model to produce contact networks, transmission networks, phylogenetic trees, and sequence datasets, and to add error to the data. FAVITES is designed to be extensible by dividing the generative model into modules, each of which is expressed as a fixed API that can be implemented using various models. We use FAVITES to simulate \gls{hiv} datasets and study the realism of the simulated datasets. We then use the simulated data to study the impact of the increased treatment efforts on epidemiological outcomes. We also study two transmission network reconstruction methods and their effectiveness in detecting fast-growing clusters.

\textit{Availability and implementation} --- FAVITES is available at \url{https://github.com/niemasd/FAVITES}, and a Docker image  can be found on DockerHub (\url{https://hub.docker.com/r/niemasd/favites}).

\section{Introduction}
The spread of many infectious diseases is driven by social and sexual networks~\cite{Kelly1991}, and reconstructing their transmission histories from molecular data may be able to enhance intervention. For example, network-based statistics for measuring the effects of \gls{art} in \gls{hiv} can yield increased statistical power~\cite{Wertheim2011}; the analysis of the growth of HIV infection clusters can yield actionable epidemiological information for disease control~\cite{Lewis2008}; transmission-aware models can be used to infer HIV evolutionary rates~\cite{Vrancken2014}.

%% END FAVITES CHAPTER