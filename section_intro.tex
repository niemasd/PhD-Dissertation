%% START INTRO CHAPTER
\addcontentsline{toc}{chapter}{\introtitle}
\chapter*{\introtitle}
\clearpage

Although they are typically associated with the study of the evolution of species, phylogenetic trees can be used to study the evolution of any sequence that evolves. For example, phylogenetic methods have been used to study the evolution of multicopy gene families~ \cite{Page1997}, cancer genomes~\cite{El-Kebir2016,Nowell1976}, antibodies~\cite{Litman1993,Robinson2015,Safonova2015}, segmental duplicates~\cite{Bailey2006,Jiang2007}, and transposable genomic elements~\cite{Dewannieux2003,Moshiri2017}, which are all entities that evolve \textit{within} the genome of a single species.

Models of tree evolution describe probability distributions over the space of tree shapes~\cite{Yule1925,Aldous2001}, which can be used as the prior distribution in a Bayesian inference~\cite{Drummond2007,Mooers2012,Sayyari2016}, to generate null distributions describing certain neutral processes~\cite{Guyer1991,Kirkpatrick1993,Agapow2002}, or to infer evolutionary parameters inherently of interest to the biologist~\cite{Morlon2014}. Similarly, generalized epidemic models describe probability distributions over the space of transmission networks~\cite{Sahneh2013}, and simulations that sample the distributions defined by these stochastic models allows epidemiologists to study infection patterns of disease epidemics~\cite{Sahneh2017}.

The spread of many infectious diseases is driven by social and sexual networks~\cite{Rivas2012}, and reconstruction of their transmission histories from molecular data can greatly enhance intervention. For example, network-based statistics for measuring \gls{HIV} treatment effects can yield increased statistical power~\cite{Wertheim2011}; the analysis of the growth of \gls{HIV} infection clusters can yield actionable epidemiological information for disease treatment and prevention~\cite{Aldous2012,Brenner2013}; transmission-aware models can be used to infer rates of \gls{HIV} evolution~\cite{Vrancken2014}. Multiple methods exist that utilize phylogenetic inference in the reconstruction of transmission networks~\cite{Prosperi2011,Ragonnet-Cronin2013,Moshiri2018b,Balaban2019}, but the accuracies, errors, and limitations of these methods are still poorly understood.

By utilizing models of viral transmission, tree evolution, and sequence evolution together, epidemiologists can simulate realistic data representative of a virus of interest as it spreads through a population of interest, and the resulting data can be used to evaluate the accuracies of transmission network inference methods as well as to study trends and patterns of an epidemic as a function of the various model parameters to gain insights into the mechanisms driving the epidemic of interest~\cite{Ratmann2017}.

%% END INTRO CHAPTER